\documentclass[a4paper,12pt,english]{article}
\usepackage{soul}
\usepackage{titling}
\usepackage[vmargin=20mm,hmargin=25mm]{geometry}
\usepackage{url}
\usepackage[utf8]{inputenc}
\usepackage{enumitem}
\usepackage[english]{babel}
\usepackage{changepage}
\usepackage{makecell}
\usepackage{bookmark}
\usepackage{isodate}
\usepackage{hyperref}
\usepackage[level]{datetime}


\usepackage{tabularx}
    \renewcommand\tabularxcolumn[1]{m{#1}}
    \newcolumntype{C}{>{\centering\arraybackslash}X}

%\setlength{\parindent}{2em}
\setlength{\parskip}{1em}

% Fixes date in section
    \newcommand{\printdatetitle}[1]{
      \texorpdfstring{\printdate{#1}}{#1}
    }

    \makeatletter
    \providecommand{\@nil}
    \makeatother
    
    \dateinputformat{tex}

\newcommand{\email}[1]{\href{mailto:#1}{\nolinkurl{#1}}}

\begin{document}

\title{Progress Log}
\author{Iris Kutsyy}
\date{\protect\today}

\maketitle
\frenchspacing

\section*{\printdatetitle{2020/10/26}}
\begin{itemize}
\item Set up git
\item Started progress log
\end{itemize}


\section*{\printdatetitle{2020/11/02}}
\begin{itemize}
\item Slower progress since the announcement of national lockdown
\item AlGhamdi et al. and Matta et al. simply measure the total runtime, I will simply measure time per node, total time, memory usage, and number of graph reads. I will use \verb|System.nanoTime|.
\item For the time being, will represent graphs by an array of nodes and set of edges
\item Began skeleton of framework
\item Attempted to get access to high power computing on Windows, failed at CL VPN setup, will try on Linux
\item Downloaded test datasets
\end{itemize}

\section*{\printdatetitle{2020/11/03}}
\begin{itemize}
\item Worked on setting up access to CL resources
\end{itemize}

\section*{\printdatetitle{2020/11/04}}
\begin{itemize}
\item Successfully got access to the high power computing servers rio, nile, and yellow. \textbf{MILESTONE}
\end{itemize}

\section*{\printdatetitle{2020/11/05}}
\begin{itemize}
\item Got access working on Linux boot as well.
\item First day of national lockdown.
\end{itemize}

\section*{\printdatetitle{2020/11/04}}
\begin{itemize}
\item Completed framework. \textbf{MILESTONE}
\end{itemize}

\section*{\printdatetitle{2020/11/14}}
\begin{itemize}
\item Completed \texttt{Brandes} for weighted and unweighted graphs. \textbf{MILESTONE}
\item Tested on unweighted graphs ONLY. (by using known graphs
\end{itemize}

\section*{\printdatetitle{2020/11/20}}
\begin{itemize}
\item Tested \texttt{Brandes} against \texttt{JGraphT}. \textbf{MILESTONE}
\item Found that I had bugs
\end{itemize}

\section*{\printdatetitle{2020/11/21}}
\begin{itemize}
\item Fixed bugs, tested on a medium (1k nodes) multigraph.
\item Added multigraph support
\end{itemize}

\section*{\printdatetitle{2020/11/30}}
\begin{itemize}
\item Completed introduction
\item Attempting to do Fibonacci Heap for Brandes weighted queue, but JGraphT has removed features
\item Switching back to JGraphT 1.3.0 because it is the last documented version (we love documentation\dots)
\item Cannot switch to older JGraphT because the way Graphs are imported has changed
\item Instead using FibHeap from here: 

\url{https://keithschwarz.com/interesting/code/?dir=fibonacci-heap}
\item Have bugs :(
\end{itemize}

\section*{\printdatetitle{2020/12/1}}
\begin{itemize}
\item Fixed bug! It was a 0-weight cycle in the input
\item Got introduction approved!
\item Will do background where I go into depth about previous work and the algorithms
\item Copying from my own proposal is fine according to Tim
\item Optimized Brandes, it is now faster than JGraphT
\end{itemize}

\section*{\printdatetitle{2020/12/4} - \printdatetitle{2020/12/11}}
\begin{itemize}
\item Attempted to use a simpler representation of graphs - SetGraph, ran tests to compare to original (running on large graph), couldn't get it faster than {\raise.2ex\hbox{$\scriptstyle\sim$}}10\% slower, even with \texttt{trove} primitive hashmaps.
\end{itemize}

\section*{\printdatetitle{2020/12/11}}
\begin{itemize}
\item Brandes++ implementation can be found at \url{https://cs-people.bu.edu/edori/code.html#Betweenness_centrality}
\item Holy shit that implementation is god awful wtf it's unreadable, I'm not gonna have trouble showing that my code is as fast as it, I would have trouble being slower than 100x faster
\item My \texttt{Brandes} implementation runs 55500\% faster\dots
\end{itemize}

\section*{\printdatetitle{2020/12/14}}
\begin{itemize}
\item Using Metis since it's consistently the fastest in Brandes++
\item Using \url{http://glaros.dtc.umn.edu/gkhome/node/78}
\item Using HEM since it's consistently the best tested by the above paper, and it is easy to revert to RM
\item Merging weighted and unweighted graphs since the difference in memory usage is small and not worth the development time (and casting nonsense)
\end{itemize}

\section*{\printdatetitle{2020/12/15}}
\begin{itemize}
\item Using KGGGP since it was found to consistently perform best for partition
\item \url{https://ieeexplore.ieee.org/document/7445341}
\item Doing 5-way partition initially
\item Using Local Greedy Approach since that's what appears to be used by METIS
\item Using GGGP as described in ``A FAST AND HIGH QUALITY MULTILEVEL SCHEME FORPARTITIONING IRREGULAR GRAPHS'' 1998
\end{itemize}
\section*{\printdatetitle{2020/12/16}}
\begin{itemize}
\item Trying with ArrayGraph and \url{https://ocw.mit.edu/courses/electrical-engineering-and-computer-science/6-895-theory-of-parallel-systems-sma-5509-fall-2003/projects/kasheff.pdf
}
\end{itemize}

\section*{\printdatetitle{2020/12/19}}
\begin{itemize}
\item The burnout got really intense, had to take a day off and it was very helpful and made me more productive today than I have been in a while.
\item Got coarsening and partitioning done
\item Using BKL(*,1) for refinement, as it was found to be the best and fastest
\item Using boundary of 2\% as in the paper.
\item Need to implement a new data structure for finding neighbors and KL refinement
\item This is sub-algorithm \#7...
\end{itemize}

\section*{\printdatetitle{2020/12/22}}
\begin{itemize}
\item Finished meletis!
\item Tested on small graphs
\item Tested that refinement reduces edge-cut
\item Tested that for all $P_i$, union $P_i$ is all nodes in G and for any $P_i, P_j$ union is empty
\item Tested partition sizes are roughly equal (not identical due to coarsening phase)
\item Manually reviewed again and again that my code matches the specification
\item URLs used:\\
\url{http://glaros.dtc.umn.edu/gkhome/metis/metis/publications}\\
``Multilevelk-way Partitioning Schemefor Irregular Graphs1'' George Karypis and Vipin Kuma\\
``MULTILEVEL GRAPH PARTITIONING SCHEMES'' George Karypis and Vipin Kumar\\
\url{https://ocw.mit.edu/courses/electrical-engineering-and-computer-science/6-895-theory-of-parallel-systems-sma-5509-fall-2003/projects/kasheff.pdf}
`` FAST AND HIGH QUALITY MULTILEVEL SCHEME FORPARTITIONING IRREGULAR GRAPHS'' GEORGE  KARYPIS†ANDVIPIN  KUMAR 1998\\
\url{https://dl.acm.org/doi/10.5555/800263.809204}\\
\url{http://users.ece.northwestern.edu/~haizhou/357/lec2.pdf}\\
\url{https://github.com/sahibjotsingh/KL-Algorithm}
\item Replaced old Brandes implementation with a primitive based approach (using ArrayGraph) - testing improvements (previous wiki-vote was 7.2 seconds, now 3.95)
\end{itemize}


\section*{\printdatetitle{2021/01/02}}
\begin{itemize}
\item Brandes++ assumes an undirected graph, need to add a djikstra's on each subgraph from each non-frontier to each frontier.
\item Paper incorrectly says looking for s$\rightarrow$v$\rightarrow$f, code and formula indicate s$\rightarrow$f$\rightarrow$v

\item Problem:
\begin{itemize}
\item In the version of the paper published on researchgate on June 2014 here: \url{https://www.researchgate.net/publication/263201045_A_Divide-and-Conquer_Algorithm_for_Betweenness_Centrality} on page 8 under the \texttt{Build\_SK} algorithm description it says ``However, \texttt{Brandes++ }is not equipped to take shortest paths starting from $s$ into consideration if $s\in V_i / F_i$ for some supernode $G_i$'' and describes that they guarantee this is not the case by making $s$ its own cluster.

\item The archivx version submitted 16 June 2014 and last edited 4 June 2015 available here:  \url{https://arxiv.org/abs/1406.4173} says on page 5 under the same section:\\

``Note  that  since we  need  to  know  the  shortest  paths  for  every  target node $s\in S$,  we  treat  the  nodes  in $S$ specially.   More specifically, given the input partition $P$, we remove all targets  from  their  respective  parts  and  add  them  as singletons.  Thus, we use the partition $<$Math$>$.  Assuming that the number of target  nodes  is  relatively  small  compared  to  the  total number of nodes in the network,  this does not have a significant effect on the running time of our algorithm.'' with no particular emphasis, immediately going to discussing how some operations could be parallelized.

\item The proceedings of the 2015 SIAM international conference on data mining says the same \url{https://epubs.siam.org/doi/10.1137/1.9781611974010.49}

\end{itemize}
\end{itemize}

\section*{\printdatetitle{2021/01/08}}
\begin{itemize}
\item So, turns out the paper is for the shortest paths between a subset of nodes, I've been struggling with that for a few days.
\item Will implement and compare to an approach using a modified Brandes
\item Their implementation takes $O(\sum{|S||F_i|\sum{|V_i / F_i|}})$ which is slower than the paper describes
\item Switched from combining ints to longs to just using a map of maps after finding that the latter is faster (intCombiningTest)
\end{itemize}


\section*{\printdatetitle{2021/01/13}}
\begin{itemize}
\item Testing on p2p unweighted undirected:\\
Binary heap: 66.85s
Fibonacci heap: 65.54s
\item Nope I'm just stupid, that was with unweighted so didn't use the heaps
\end{itemize}

\section*{\printdatetitle{2021/01/15}}
\begin{itemize}
\item Implemented RankPairingHeap and it works!
\end{itemize}

\section*{\printdatetitle{2021/01/27}}
\begin{itemize}
\item Done with Geisberger!
\item Also implemented GeisbergerLinear, GeisbergerEnumerative, and Brandes(2008)
\end{itemize}
\end{document}
