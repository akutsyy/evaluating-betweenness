\documentclass[a4paper,12pt]{article}
\usepackage{soul}
\usepackage{titling}
\usepackage[vmargin=20mm,hmargin=25mm]{geometry}
\usepackage{url}
\usepackage[utf8]{inputenc}
\usepackage{enumitem}
\usepackage[english]{babel}
\usepackage{changepage}
\usepackage{makecell}
\usepackage{hyperref}


\usepackage{tabularx}
    \renewcommand\tabularxcolumn[1]{m{#1}}
    \newcolumntype{C}{>{\centering\arraybackslash}X}

%\setlength{\parindent}{2em}
\setlength{\parskip}{1em}

\renewcommand{\baselinestretch}{0.99}

\setlength{\droptitle}{-5em}   % This is your set screw

\newcommand{\email}[1]{\href{mailto:#1}{\nolinkurl{#1}}}

\begin{document}

\title{{\Large Computer Science Tripos - Part II - Project Progress Report}\vspace{0.5cm}\\
Evaluating Betweenness Centrality Algorithms for Real World Datasets}

\date{\parbox{\linewidth}{\centering% 
\vspace{-2cm}     
Iris Kutsyy \hspace{1cm} ak2149@cam.ac.uk \hspace{1cm} Trinity College \\ \vspace{3mm}  
      \today\endgraf\vspace{3mm}
  }}

\maketitle
\frenchspacing

\noindent
\textbf{Project Supervisor:} Dr. Timothy Griffin

\noindent
\textbf{Directors of Studies:} Dr. Sean Holden, Dr. Neel Krishnaswami, Prof. Frank Stajano

\noindent
\textbf{Project Overseers:} Prof. Marcelo Fiore, Dr. Robert Mullins

\section*{Progress Report}

Despite some unexpected difficulty, my Part II project is exactly on pace. My project requires implementing five algorithms, instrumenting them with a variety of metrics, and evaluating their performance on large graphs.

I quickly researched instrumentation and decided on a set of metrics for evaluating the performance of the algorithms (time spent per node, total time, memory usage, and the number of graph reads). By November 4th I also successfully logged onto the high power computing server I was granted access to, meeting my milestones.


By November 20th I created a framework for running and testing the centrality algorithms, and implemented my first algorithm, the \texttt{Brandes} algorithm. I verified its output using \texttt{JGraphT}, a Java graph analysis library.

I then optimized my graph representation and \texttt{Brandes} implementation, and ended up with an implementation that runs about 10 times faster than the \texttt{JGraphT} implementation.

I ran my implementation of \texttt{Brandes} on the high performance server to determine the maximum size of graphs I can use for my experimentation. I realized that my laptop actually has higher single threaded performance than the server, so will use it instead.

Following this, I began implementing the \texttt{Brandes++} algorithm. Difficulty arose because that algorithm requires implementing a specific (and very complicated) graph clustering algorithm to achieve the performance the paper describes. That graph algorithm itself requires implementing three more algorithms and a new data structure. I finished the graph partition algorithm by the 22nd of December, missing my milestone for finishing \texttt{Brandes++}. However, I'd set aside a buffer for unforseen difficulty and finished implementing and testing the \texttt{Brandes++} algorithm by January 12th.

By January 20th I finished the algorithm by Geisberger et al, meeting my milestone.

I additionally implemented multiple heap algorithms (Rank-Pair Heap, Binary Heap, Fibonacci Heap) to determine which allowed the \texttt{Brandes} algorithm to run fastest (It turned out to be the binary heap, which I'd already been using).

I have fully read through the remaining two algorithms and neither have complications like \texttt{Brandes++} does, so I expect to finish my project on time.

\end{document}
